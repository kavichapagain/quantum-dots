%%%%%%%%%%%%%%%%%%%%%%%%%%%%%%%%%%%%%%%%%%%
%%% DOCUMENT PREAMBLE %%%
\documentclass[12pt]{report}
\usepackage[english]{babel}
%\usepackage{natbib}
\usepackage{url}
\usepackage{amsmath}
\usepackage[utf8x]{inputenc}
\usepackage{amsmath}
\usepackage{graphicx}
\graphicspath{{images/}}
\usepackage{parskip}
\usepackage{fancyhdr}
\usepackage{vmargin}
\setmarginsrb{3 cm}{2.5 cm}{3 cm}{2.5 cm}{1 cm}{1.5 cm}{1 cm}{1.5 cm}
\newcommand*{\hatH}{\hat{\mathcal{H}}}

\title{Quantum Dots}								
% Title
\author{ Kavi Chapagain}						
% Author
\date{Experiment 3}
% Date

\makeatletter
\let\thetitle\@title
\let\theauthor\@author
\let\thedate\@date
\makeatother

\pagestyle{fancy}
\fancyhf{}
\rhead{\theauthor}
\lhead{\thetitle}
\cfoot{\thepage}
%%%%%%%%%%%%%%%%%%%%%%%%%%%%%%%%%%%%%%%%%%%%
\begin{document}

%%%%%%%%%%%%%%%%%%%%%%%%%%%%%%%%%%%%%%%%%%%%%%%%%%%%%%%%%%%%%%%%%%%%%%%%%%%%%%%%%%%%%%%%%

\begin{titlepage}
	\centering
    \vspace*{0.5 cm}
   % \includegraphics[scale = 0.075]{bsulogo.png}\\[1.0 cm]	% University Logo
\begin{center}    \textsc{\Large   Modern Physics}\\[2.0 cm]	\end{center}% University Name
	\textsc{\Large PEN231  }\\[0.5 cm]				% Course Code
	\rule{\linewidth}{0.2 mm} \\[0.4 cm]
	{ \huge \bfseries \thetitle}\\
	\rule{\linewidth}{0.2 mm} \\[1.5 cm]
	
	\begin{minipage}{0.4\textwidth}
		\begin{flushleft} \large
		%	\emph{Submitted To:}\\
		%	Name\\
          % Affiliation\\
           %contact info\\
			\end{flushleft}
			\end{minipage}~
			\begin{minipage}{0.4\textwidth}
            
			\begin{flushright} \large
			\emph{Submitted By :} \\
			Kavi Chapagain  
		\end{flushright}
           
	\end{minipage}\\[2 cm]
	
    
    
	
\end{titlepage}

%%%%%%%%%%%%%%%%%%%%%%%%%%%%%%%%%%%%%%%%%%%%%%%%%%%%%%%%%%%%%%%%%%%%%%%%%%%%%%%%%%%%%%%%%

\tableofcontents
\pagebreak

%%%%%%%%%%%%%%%%%%%%%%%%%%%%%%%%%%%%%%%%%%%%%%%%%%%%%%%%%%%%%%%%%%%%%%%%%%%%%%%%%%%%%%%%%
\renewcommand{\thesection}{\arabic{section}}
\section{Introduction}

Quantum Dots are the real world particle in a box. They are the nanoparticles in which a a particular colour is observed when illuminated by the light. It has a lot of applications ranging from using in a solar cells to the quantum screen used in computer/TV screens.
\newline In this experiment we are determining the size of a Quantum Dot. First the formula for three dimensional energy equation is derived then the size of the Quantum Dot is determined.


\section{Theory}
In 1925 Austrian Physicist Erwin Shrondinger introduced the wave equation, named after him Shrondinger Equation. The equation is:
\begin{equation}
    \hatH \Psi = E \Psi          
\end{equation}

Here, $\hatH$ is the Hamiltonian operator is the eigen value of the wave function$\psi$. The $\hatH$ can be expressed as:
\begin{equation}
    \hatH = - \frac{\hbar^2}{2m} \frac{\delta^2}{\delta x^2} + V(x)
\end{equation}
 From equation 1 and equation 2:
 \begin{equation}
    - \frac{\hbar^2}{2m} \frac{\delta^2 \Psi(x)}{\delta x^2} + V(x) = E \Psi(x)
 \end{equation}
First we will derive the energy for one dimensional particle in a box then derive three dimensional later. Since the potential inside the well is zero, the Equation 3 can be simplified as:
\begin{equation}
     - \frac{\hbar^2}{2m} \frac{\delta^2 \Psi(x)}{\delta x^2} = E \Psi(x)
\end{equation}
Rearranging Equation 4,
\begin{equation}
    \frac{\delta^2 \Psi(x)}{\delta x^2} =  - \frac{2mE}{\hbar^2} \Psi(x)
\end{equation}
Define a variable $k^2$:
\begin{equation}
    k^2 = \frac{2mE}{\hbar}
\end{equation}
Now, the Shrondinger Equation is further simplified as:
\begin{equation}
    \frac{\delta^2 \Psi(x)}{\delta x^2} =  -k^2 \Psi(x)
\end{equation}
The general solution of this differential equation is
\begin{equation}
    \Psi(x) = Asin(kx) + Bcos(kx)
\end{equation}
and,
\begin{equation}
      E = \frac{k^\hbar}{2m}
\end{equation}

The potential outside the box is infinite, the wave function should be 0 at the edge of the box. 
\begin{equation}
    \Psi(0) = \Psi(L) = 0
\end{equation}

So, when x=0,
\begin{equation}
    \Psi(0) = Asin(0) + Bcos(0) = A.0 + B.1 = 0
\end{equation}
This means the value of B is 0 at the boundary. So, our wave function becomes
\begin{equation}
    \Psi(x) = Asin(kx)
\end{equation}
Now, when x=L,
\begin{equation}
    \Psi(L) = Asin(kL) = 0
\end{equation}
The Equation 12 implies
\begin{equation}
    kL = \pi, 2\pi, 3\pi, ...
\end{equation}
which means
\begin{equation}
    k = \frac{n\pi}{L}, n=1, 2, 3, ...
\end{equation}
From Equation 10 and Equation 14, the wave function is:
\begin{equation}
    \Psi_n(x) = Asin(\frac{n\pi x}{L})
\end{equation}
and
\begin{equation}
    E_n = \frac{n^2\pi^2\hbar^2}{2mL^2}
\end{equation}
This is for the one dimensional square shape particle. \newpage 
For the three dimensional and spherical particle which is the shape of the Quantum Dots used in this experiment. The lowest energy state is given by 

\begin{equation}
    E_{sphere} = \frac{\hbar^2 \pi^2}{2mR^2},
\end{equation}
where R is the radius of the Quantum Dot
\newline
There are two particles, the electron and the hole, so the energy is
\begin{equation}
    E_{sphere} = \frac{\hbar^2 \pi^2}{2m_eR^2} + \frac{\hbar^2 \pi^2}{2m_hR^2}, \label{eq: 3d-qd}
\end{equation}
In the Equation \eqref{eq: 3d-qd}, $m_e$ is the mass of the electron, $m_h$ is the mass of the hole inside the semiconductor and R is the radius of the Quantum Dot.
\newline
The Quantum Dot box is filled with a semiconductor. Taking that in account, the energy of semiconductor band gap ($E_g$) is added our base-line energy. Therefore, 
\begin{equation}
     E_{sphere} = \frac{\hbar^2 \pi^2}{2m_eR^2} + \frac{\hbar^2 \pi^2}{2m_hR^2} + E_g 
\end{equation}
The following values are available:
\newline
$E_g = 2.15 \times 10^-19$ J
\newline
$E_g = 7.29 \times 10^-32$ kg
\newline
$E_g = 5.47 \times 10^-31$ kg

Similarly, in the experiment, the wavelength of the light from the Quantum Dots is measured and the zero-point energy is calculated by using
\begin{equation}
    E = hf \: and \: f = \frac{c}{\lambda}
\end{equation}
where E, h, f, c and $\lambda$ are is the energy, Plank's constant(6.625$\times 10^-34$Js), frequency, speed of the light (3.0$\times 10^8$m/s) and the wavelength respectively.


\section{Experiment}
The experiment is conducted to determine the radii of four different size of Quantum Dots. The four quantum dots were in a different solution of different colour, red, orange, yellow and green. A clamp was set to hold the solution and were exposed with 400nm LED light. Then the data was collected by using spectrometer sensor and a computer program Data Logger.
\newline
table
\begin{center}
\begin{tabular}{ |c c c| } 
\hline
 \# & Colour & $\lambda(nm)$\\
\hline
1 & Red & 612.7\\
2 & Orange & 592.2 \\ 
3 & Yellow & 565.7 \\ 
4 & Green & 537.1 \\ 
\hline
\end{tabular}
\end{center}


\section{Analysis}

\section{Conclusion}






 
\begin{thebibliography}{111}
   
  \bibitem{ACMT}
A. Aldroubi, C. Cabrelli, U. Molter, and Sui Tang,
Dynamical sampling, 
{\it  Applied and Computational Harmonic Analysis}, doi:10.1016/j.acha.2015.08.014, 2016

%if the "underfill \hbox" warning bothers you uncomment the following line
%\raggedright
\bibitem{ACAMP}
    A. Aldroubi, C. Cabrelli, A. F. Cakmak, U. Molter,  and A. Petrosyan,
    Iterative actions of normal operators, 
    Submitted. Available at http://arxiv.org/abs/1602.04527.
  
\bibitem{Gro01} 
    K. Groechenig,
    {\it Foundations of time-frequency analysis}, 
    Birkh\"auser Boston, 2001.

\end{thebibliography}
\end{document}

